% Document class : AMS Journal
\documentclass{jams-l}

% Package
\usepackage{mathpazo}
\usepackage{euler}
\usepackage{tikz-cd}
\usepackage{url}
\usepackage{amssymb}
\usepackage{mathtools}

% Theorem & Style
\newtheorem{theorem}{Theorem}[section]
\newtheorem{lemma}[theorem]{Lemma}
\theoremstyle{definition}
\newtheorem{definition}[theorem]{Definition}
\newtheorem{example}[theorem]{Example}
\theoremstyle{remark}
\newtheorem{remark}[theorem]{Remark}
\theoremstyle{corollary}
\newtheorem{corollary}[theorem]{Corollary}
\theoremstyle{claim}
\newtheorem{claim}[theorem]{Claim}

%    Exercise & Solution 
\newtheorem*{xca*}{Exercise}
\newenvironment{solution}
{\begin{proof}[Solution]}
{\end{proof}}
\newenvironment{exercise}[1]
{\begin{xca*}[$\mathbf{#1}$] %\phantom{a}
}
{\end{xca*}}


\numberwithin{equation}{section}

%    Math Notation
\newcommand{\abs}[1]{\lvert#1\rvert}
\newcommand{\Ar}{\mathsf{Ar}}
\newcommand{\Ab}{\mathsf{Ab}}
\newcommand{\Set}{\mathsf{Set}}
\newcommand{\Hom}{\mathsf{Hom}}
\newcommand{\Mor}{{Mor}}
\newcommand{\Top}{\mathfrak{Top}}
\newcommand{\Rmod}{R\text{-}\mathsf{mod}}
\newcommand{\modR}{\mathsf{mod}\text{-}R}
\newcommand{\Spec}{\mathrm{Spec}}

\DeclareMathOperator{\Psh}{Psh}
\DeclareMathOperator{\Sh}{Sh}
\DeclareMathOperator{\im}{im}
\DeclareMathOperator{\coker}{coker}
\DeclareMathOperator{\op}{op}
\DeclareMathOperator{\id}{\mathsf{id}}
\DeclareMathOperator{\Div}{Div}
\DeclareMathOperator{\PrDiv}{PrDiv}
\DeclareMathOperator{\ord}{ord}
\DeclareMathOperator{\Pic}{Pic}

%    Blank box placeholder for figures (to avoid requiring any
%    particular graphics capabilities for printing this document).
\newcommand{\blankbox}[2]{%
  \parbox{\columnwidth}{\centering
%    Set fboxsep to 0 so that the actual size of the box will match the
%    given measurements more closely.
    \setlength{\fboxsep}{0pt}%
    \fbox{\raisebox{0pt}[#2]{\hspace{#1}}}%
  }%
}

\begin{document}

\title{THE RISING SEA : Foundations of Algebraic Geometry}

%    Information for the first author
\author{Sang min Ko}
\address{Department of Mathematics, Columbia University}
\email{smko77@math.columbia.edu}

%    Information for the second author
%\author{Person1}
%\address{Abcd University}
%\email{abcd@mail.com}

\dedicatory{This is the solution of the problem sets in FOAG}

\maketitle

% When you use multiple equations at the same time, I strongly recommend to use 
% \begin{equation*}
% \begin{aligned}
% asdf & = adf \\
% adf  & = asdf 
% \end{aligned}
% \end{equation*}

\section{Dimension and codimension}
\begin{exercise}{11.1.A} 
\end{exercise}
\begin{solution}
It follows from the correspondence between
\[ \{ \text{prime ideals of }A \} \Longleftrightarrow \{ \text{irreducible closed susbet of } \Spec A\} \] 
\end{solution}

\begin{exercise}{11.1.B}
\end{exercise}
\begin{solution}
It follows from the bijection
\end{solution}


\section{Dimension, transcendence degree and Noether normalization}
\begin{exercise}{11.2.A}
\end{exercise}
\begin{solution}
Refer to Milne's note
\end{solution}



\section{Codimension one miracle}
\begin{exercise}{11.3.A}

\end{exercise}




\bibliographystyle{amsplain}
\begin{thebibliography}{10}

\bibitem {FOAG} R. Vakil, \textit{The Rising Sea : Foundation of Algebraic Geometry}, math216.wordpress.com, 2017.

\end{thebibliography}

\end{document}

%------------------------------------------------------------------------------
% End of jams_sample.tex
%------------------------------------------------------------------------------